\newpage
{\color{gray}\hrule}
\begin{center}
\section{Deducing the Differential Equation}
\bigskip
\end{center}
{\color{gray}\hrule}
\begin{multicols}{2}

\subsection{Defining the Square Wave}
The square wave \(v(t)\) alternates between two values, \(V_0\) and 0, with a period \(T\). The duty ratio \(\alpha\) determines the fraction of the period during which the voltage is \(V_0\). Specifically:
\begin{itemize}
    \item For \(0 \leq t < \alpha T\), \(v(t) = V_0\).
    \item For \(\alpha T \leq t < T\), \(v(t) = 0\).
\end{itemize}

\subsection{The Differential Equation for a Series RL Circuit}
The voltage across a series RL circuit is given by Kirchhoff's Loop Law as:
\[
v(t) = L \frac{di(t)}{dt} + R i(t)
\]
where \(i(t)\) is the current through the circuit.

\subsection{Solving the Differential Equation}
We need to solve this equation for two intervals:
\begin{enumerate}
    \item When \(v(t) = V_0=10\) (for \(0 \leq t < \alpha T\)):
    \begin{align}
        V_0 = L \frac{di(t)}{dt} + R i(t)\\
        \implies\frac{di(t)}{dt}+\frac{R}{L}i(t) =\frac{10}{L}
    \end{align}
    Integrating Factor $= e^{\int \frac{R}{L}dt}\implies e^{\frac{Rt}{L}}$\\
    Multiply both sides by the integrating factor and integrate.\\
    \begin{align}
        \int d(e^{\frac{Rt}{L}}i(t))&=\int e^{\frac{Rt}{L}}\frac{10}{L}dt\\
        e^{\frac{Rt}{L}}i(t) &= \frac{10}{L}\frac{e^{\frac{Rt}{L}}}{\frac{R}{L}} + C
    \end{align}
    Divide both sides with $e^{\frac{Rt}{L}}$
    \begin{align}
        i(t)&=\frac{10}{R} +Ce^{-\frac{Rt}{L}} \label{eq5}
    \end{align}
    Apply initial condition  $i(0)$ at $t=0$\\
    \begin{align}
        i(0)&=\frac{10}{R} +C\\
        C&=i(0)-\frac{10}{R} \label{eq7}
    \end{align}
    equation (\ref{eq7}) in (\ref{eq5}),
    \begin{align}
       i(t) &= \frac{10}{R} \left(1 - e^{-\frac{R}{L}t}\right) + i(0) e^{-\frac{R}{L}t}\\
       \text{since $i(0)=0$}\\
       i(t) &= \frac{10}{R} \left(1 - e^{-\frac{R}{L}t}\right)
    \end{align}
    \item When \(v(t) = 0\) (for \(\alpha T \leq t < T\)):
    \begin{align}
        0 = L \frac{di(t)}{dt} + R i(t)
    \end{align}
    The solution to this DE is:
    \begin{align}
        i(t) = i(\alpha T) e^{-\frac{R}{L}(t - \alpha T)}
    \end{align}
    where \(i(\alpha T)\) is the current at \(t = \alpha T\).
\end{enumerate}

\end{multicols}
